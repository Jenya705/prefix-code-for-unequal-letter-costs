\subsection{main.rs}
\begin{lstlisting}
#![feature(get_mut_unchecked)]

use std::{fs::File, io::BufReader};

use clap::Parser;
use input::Input;
use scanner::Scanner;
use solver::try_solve;

mod huffman;
mod input;
mod scanner;
mod solver;
mod tree;

#[derive(Parser)]
struct App {
    #[arg(short, long)]
    /// Das andere Format der Datei
    occurences: bool,
    #[arg(short, long, default_value_t = 0.0)]
    /// Um wie viel müssen Wahrscheinlichkeiten multipliziert werden (in einigen Fällen kann es einen besseren PC ausgeben)
    /// 
    /// Falls der Wert 0 ist, dann werden o_i Werte anstatt Wahrscheinlichkeiten benutzt.
    mul: f64,
    #[arg(short, long, default_value_t = false)]
    /// Ob der Huffman-Algorithmus benutzt wird.
    huffman: bool,
    #[arg()]
    /// Der Dateiname
    file_name: String,
}

fn main() {
    let app = App::parse();

    let input = Input::read(
        !app.occurences,
        app.mul,
        &mut Scanner::new(BufReader::new(File::open(app.file_name).unwrap())),
    );

    try_solve(
        &input.letters,
        &input.probabilities,
        &input.symbols,
        if app.mul == 0.0 { input.sum } else { app.mul },
        app.huffman,
    );
}
\end{lstlisting}
\subsection{solver.rs}
\begin{lstlisting}
use std::time::Instant;

use good_lp::{
    Expression, ProblemVariables, ResolutionError, Solution, SolverModel, Variable,
    VariableDefinition,
};

use crate::{
    huffman::{huffman_optimal, huffman_solve},
    tree::{adjust_costs, build_prefixes},
};

/// der in der Doku beschriebene Wert L
const LIMIT: u32 = u32::MAX;

pub struct VCacher {
    values: Vec<u32>,
}

impl VCacher {
    pub fn new(letters: &[u32], m: u32, n: u32) -> Self {
        let mut values = vec![1];
        'o: for i in 1usize..=(m as usize) {
            let mut v = 0u32;
            for &letter in letters {
                if let Some(&to_add) = values.get(i - letter as usize) {
                    match v.checked_add(to_add) {
                        Some(nv) => v = nv,
                        None => break 'o,
                    }
                }
            }
            if v > LIMIT || values[i - 1] * n > LIMIT {
                break;
            }
            values.push(v);
        }
        Self { values }
    }

    pub fn m(&self) -> u32 {
        self.values.len() as u32
    }

    pub fn get(&self, i: usize) -> u32 {
        self.values[i]
    }
}

pub fn calculate_max_lengths(cost: f64, probabilities: &[f64], m: u32, lengths: &mut [u32]) {
    let mut f = 0.0;
    let mut s = probabilities.iter().sum::<f64>();
    for i in 0..lengths.len() {
        let mut l = 1;
        let mut r = m;

        // wenn i = 0, dann ist i - 1 = -1
        if i != 0 {
            f += probabilities[i - 1];
            s -= probabilities[i - 1];
        }
        
        // f = p_1 + ... + p_{i-1}
        // s = p_i + ... + p_n

        // Binäresuche in (de facto) einer geordneten Liste, wobei die Elemente gleich
        // den Ausdruck aus der Sektion über Optimierung vom ILP-Problem sind.
        while r - l > 1 {
            let m = (l + r) / 2;
            let c = f + s * (m as f64);
            if c >= cost {
                r = m;
            } else {
                l = m;
            }
        }

        // die obere Grenze ist gebraucht (also r)
        // (l ist die untere Grenze)
        lengths[i] = r;
    }
}

pub fn try_solve(
    letters: &[u32],
    probabilities: &[f64],
    symbols: &[(char, u64)],
    mul: f64,
    use_huffman: bool,
) {
    let timer = Instant::now();

    let mut lengths = vec![0; probabilities.len()];

    if use_huffman {
        println!("huffman optimal: {}", huffman_optimal(letters));
        huffman_solve(letters, probabilities, &mut lengths);
        // kann den PC noch optimieren
        lengths.sort_unstable();
    } else {
        let mut adjusted_costs = vec![0; probabilities.len()];
        adjust_costs(letters, &mut adjusted_costs);

        let m = adjusted_costs.last().cloned().unwrap();
        let v_cacher = VCacher::new(letters, m, probabilities.len() as u32);
        let m = m.min(v_cacher.m());

        // Kalkuliert Kosten des PC aus den gegebenen Längen der Präfixe
        let calc_costs = |costs: &[u32]| -> f64 {
            costs
                .iter()
                .zip(probabilities.iter())
                .map(|(&f, &s)| f as f64 * s)
                .sum::<f64>()
        };

        // Es kann theoretisch sein, dass der von der Bestimmung des m PC besser als vom Huffman ist.
        let mut best_cost = calc_costs(&adjusted_costs);
        let mut best_adjusted_costs = adjusted_costs.clone();

        let mut costs = vec![0; probabilities.len()];
        huffman_solve(letters, probabilities, &mut costs);
        costs.sort_unstable();
        let cost = calc_costs(&costs);
        // Wählen der beste PC von den beiden
        if cost < best_cost {
            best_cost = cost;
            best_adjusted_costs = costs;
        }

        println!("best pre-ilp found cost: {}", best_cost / mul);

        let mut theoretical_max_lengths = vec![0; probabilities.len()];

        calculate_max_lengths(best_cost, probabilities, m, &mut theoretical_max_lengths);

        let result = ilp_solve(
            &v_cacher,
            letters,
            m,
            probabilities,
            &theoretical_max_lengths,
            &mut lengths,
        );

        match result {
            // ResolutionError::Infeasible meint dass es keine Lösung gefunden wurde, weil
            // die Lösung (im Fall) anhand der Optimierung des Problems nicht zu finden ist.
            Ok(_) | Err(ResolutionError::Infeasible) => {}
            Err(ref err) => println!("ILP Solver error: {:?}", err),
        }

        // Entweder ein Fehler oder der PC, der durch Huffman gefunden wurde, ist besser
        if result.is_err() || calc_costs(&lengths) > best_cost {
            lengths = best_adjusted_costs;
            println!("using already calculated result as the ilp solver didn't find any better possibilities");
        }
    }

    let mut prefixes = vec![vec![]; probabilities.len()];
    println!(
        "avg cost: {}, cost: {}",
        probabilities
            .iter()
            .zip(lengths.iter())
            .map(|(&f, &s)| f / mul * s as f64)
            .sum::<f64>(),
        symbols
            .iter()
            .zip(lengths.iter())
            .map(|(f, &s)| f.1 * s as u64)
            .sum::<u64>(),
    );
    println!("lengths: {lengths:?}");
    build_prefixes(letters, &lengths, &mut prefixes);
    for i in 0..probabilities.len() {
        println!("'{}': {:?}", symbols[i].0, prefixes[i]);
    }

    println!("time elapsed: {:?}", timer.elapsed());
}

pub fn ilp_solve(
    v: &VCacher,
    _letters: &[u32],
    m: u32,
    probabilities: &[f64],
    max_lengths: &[u32],
    lengths: &mut [u32],
) -> Result<(), good_lp::ResolutionError> {
    let n = probabilities.len() as u32;

    // Enthält alle Variablen und ihre Konfigurationen
    let mut problem = ProblemVariables::new();
    
    // Enthält Referenzen zu den Variablen
    let mut variables = vec![vec![Option::<Variable>::None; m as usize]; n as usize];

    for i in 0..n {
        for j in 0..(max_lengths[i as usize].min(m)) {
            variables[i as usize][j as usize] =
                Some(problem.add(VariableDefinition::new().binary()));
        }
    }

    // Der Ausdruck, der minimiert sein muss
    let mut minimise = Expression::default();
    for i in 0..n {
        for j in 0..m {
            if let Some(var) = variables[i as usize][j as usize] {
                minimise = minimise + probabilities[i as usize] * ((j + 1) as f64) * var;
            }
        }
    }

    let mut problem = problem
        .minimise(minimise)
        .using(good_lp::default_solver);

    // Die erste Einschänkungen
    for i in 1..m {
        let mut expr = Expression::default();
        for p in 1..=i {
            for j in 1..=n {
                if let Some(var) = variables[j as usize - 1][p as usize - 1] {
                    expr = expr + var * v.get((i - p) as usize);
                }
            }
        }
        problem = problem.with(expr.leq(v.get(i as usize)));
    }

    // Die zweite Einschränkungen
    for i in 0..n {
        let mut expr = Expression::default();
        for j in 0..m {
            if let Some(var) = variables[i as usize][j as usize] {
                expr = expr + var;
            }
        }
        problem = problem.with(expr.eq(1));
    }

    let res = problem.solve()?;

    for i in 0..n {
        for j in 0..m {
            // >= 0.5, weil die Vergleichoperationen bei Floats nicht sehr sicher sind.
            if res.value(variables[i as usize][j as usize].unwrap()) >= 0.5 {
                lengths[i as usize] = j + 1;
                break;
            }
        }
    }

    lengths.sort_unstable();

    Ok(())
}
\end{lstlisting}
\subsection{huffman.rs}
\begin{lstlisting}
use std::{cmp::Ordering, collections::BinaryHeap, rc::Rc};

/// Gibt zurück, ob der Huffman-Algorithmus für den Fall optimal wird.
pub fn huffman_optimal(letters: &[u32]) -> bool {
    letters.iter().all(|&v| v == letters[0])
}

/// Eine Knotenstructur
#[derive(Debug)]
struct Node {
    probability: f64,
    parent: Option<Rc<Node>>,
    /// Länge (bzw. Kosten) zwischen dem Elternknoten und diesem Knoten  
    length: u32,
}

impl PartialEq for Node {
    fn eq(&self, other: &Self) -> bool {
        self.probability == other.probability
    }
}

impl PartialOrd for Node {
    fn partial_cmp(&self, other: &Self) -> Option<Ordering> {
        other.probability.partial_cmp(&self.probability)
    }
}

impl Eq for Node {}

impl Ord for Node {
    fn cmp(&self, other: &Self) -> Ordering {
        self.partial_cmp(other).unwrap()
    }
}

pub fn huffman_solve(letters: &[u32], probabilities: &[f64], lengths: &mut [u32]) {
    let n = probabilities.len() % (letters.len() - 1);
    let n = if n == 0 { letters.len() - 1 } else { n };

    let mut heap = BinaryHeap::new();
    // Alle Blätter
    let mut original = Vec::with_capacity(probabilities.len());

    for &probability in probabilities.iter() {
        let node = Rc::new(Node {
            probability,
            parent: None,
            length: 0,
        });

        original.push(Rc::clone(&node));
        heap.push(node);
    }

    // l ist die Anzahl der Kinder
    let mut create_node = |l: usize| -> bool {
        // root node
        // am Ende bleibt der Kernknoten immer
        if heap.len() == 1 {
            return false;
        }

        let mut parent = Rc::new(Node {
            probability: 0.0,
            parent: None,
            length: 0,
        });

        for i in (0..l).rev() {
            let mut node = heap.pop().unwrap();
            // SAFETY:
            // - no lifetimes
            // - no raw references to any rc exist
            unsafe { Rc::get_mut_unchecked(&mut parent) }.probability += node.probability;
            unsafe { Rc::get_mut_unchecked(&mut node) }.length = letters[i];
            unsafe { Rc::get_mut_unchecked(&mut node) }.parent = Some(Rc::clone(&parent));
        }

        heap.push(parent);

        true
    };

    // Zuerst ein Knoten mit n Kindern
    if n != 1 {
        create_node(n);
    }
    // Danach mit r Kindern
    while create_node(letters.len()) {}

    // Kalkulieren der a_i Werte
    for (i, node) in original.iter().enumerate() {
        let mut current = node;
        // Gehen vom Blatt zum Kernknoten
        while let Some(ref parent) = current.parent {
            lengths[i] += current.length;
            current = parent;
        }
    }
}
\end{lstlisting}
\subsection{tree.rs}
\begin{lstlisting}
use std::{
    cmp::Reverse,
    collections::{BinaryHeap, HashMap},
};

/// In der Sektion "Finden eines zufälligen Codes" beschriebene Funktion.
/// adjusted muss q Werten enthalten.
/// Die Funktion schreibt dann in die Liste die Ergebniswerte.
pub fn adjust_costs(letters: &[u32], adjusted: &mut [u32]) {
    let mut heap = BinaryHeap::new();
    heap.push(Reverse(0u32));
    for i in 0..adjusted.len() {
        loop {
            let node = heap.pop().unwrap().0;
            if node < adjusted[i] || (heap.is_empty() && i != adjusted.len() - 1) {
                for k in 0..letters.len() {
                    heap.push(Reverse(letters[k] + node));
                }
            } else {
                adjusted[i] = node;
                break;
            }
        }
    }
}

pub fn build_prefixes(letters: &[u32], costs: &[u32], prefixes: &mut [Vec<usize>]) {
    /// Implementiert Vergleichfunktionen,
    /// so dass die Liste ignoriert wird und der binäre Heap Min Heap war
    struct InHeap(u32, Vec<usize>);

    impl PartialEq for InHeap {
        fn eq(&self, other: &Self) -> bool {
            self.0 == other.0
        }
    }

    impl PartialOrd for InHeap {
        fn partial_cmp(&self, other: &Self) -> Option<std::cmp::Ordering> {
            other.0.partial_cmp(&self.0)
        }
    }

    impl Eq for InHeap {}

    impl Ord for InHeap {
        fn cmp(&self, other: &Self) -> std::cmp::Ordering {
            self.partial_cmp(other).unwrap()
        }
    }

    // Ein Max-Heap
    let mut heap = BinaryHeap::new();
    heap.push(InHeap(0u32, vec![0usize; 0]));
    for i in 0..costs.len() {
        loop {
            let InHeap(node, prefix) = heap.pop().unwrap();
            // nicht notwendig
            assert!(node <= costs[i]);
            if node == costs[i] {
                prefixes[i] = prefix;
                break;
            } else {
                // der Knoten gehört keinem Symbol, also muss er erweitert werden
                for i in 0..letters.len() {
                    let mut prefix = prefix.clone();
                    prefix.push(i);
                    heap.push(InHeap(letters[i] + node, prefix));
                }
            }
        }
    }
}

/// Die Funktion optimiert die Präfixe, damit sie die verkürzt, wenn es möglich ist
#[allow(unused)]
pub fn optimize_prefixes(prefixes: &mut [Vec<usize>]) {
    let mut map = HashMap::new();

    for prefix in prefixes.iter() {
        for i in 1..prefix.len() {
            let key = &prefix[..i];
            if map.contains_key(key) {
                *map.get_mut(key).unwrap() += 1;
            } else {
                map.insert(key.to_vec(), 1);
            }
        }
    }

    for prefix in prefixes {
        loop {
            let l = prefix.len() - 1;
            if matches!(map.get(&prefix[0..l]), Some(&1)) {
                prefix.remove(l);
            } else {
                break;
            }
        }
    }
}

/// Kalkuliert Kosten des PC mit den gegebenen Präfixen
#[allow(unused)]
pub fn calculate_costs(letters: &[u32], prefixes: &[Vec<usize>], costs: &mut [u32]) {
    for i in 0..prefixes.len() {
        costs[i] = 0;
        for &letter in &prefixes[i] {
            costs[i] += letters[letter];
        }
    }
}
\end{lstlisting}
\subsection{input.rs}
\begin{lstlisting}
use std::collections::HashMap;

use crate::scanner::Scanner;

pub struct Input {
    pub letters: Vec<u32>,
    pub probabilities: Vec<f64>,
    pub symbols: Vec<(char, u64)>,
    pub sum: f64,
}

impl Input {
    pub fn read(message: bool, mut mul: f64, scanner: &mut Scanner<impl std::io::BufRead>) -> Self {
        let r = scanner.read::<usize>();
        let mut letters = vec![];
        for _ in 0..r {
            letters.push(scanner.read());
        }

        letters.sort_unstable();

        let mut symbols = vec![];
        let mut probabilities = vec![];

        // Es gibt zwei Möglichkeiten vom Dateiformat:
        // - eine reine Nachricht
        // - eine Liste von Wahrscheinlichkeiten
        if message {
            let msg = scanner.read_line();
            let mut map = HashMap::new();
            for c in msg.chars() {
                *map.entry(c).or_insert(0) += 1;
            }
            symbols = map.into_iter().collect::<Vec<_>>();
        } else {
            for _ in 0..scanner.read::<usize>() {
                let c = scanner.read::<char>();
                let p = scanner.read::<u64>();
                symbols.push((c, p));
            }
        }
        let sum = symbols.iter().map(|&(_, v)| v).sum::<u64>() as f64;
        // sortieren in absteigender Reihenfolge (größere Werte zuerst)
        symbols.sort_unstable_by_key(|&(_, v)| u64::MAX - v);
        // wenn mul gleich sum ist, dann werden alle Wahrscheinlichkeiten gleich 
        // die Anzahl der Symbole in der Nachricht (bzw. die von Datei gegebene Wahrscheinlichkeiten)
        if mul == 0.0 {
            mul = sum;
        }
        for &(_, v) in symbols.iter() {
            probabilities.push(v as f64 * (mul / sum));
        }

        Self {
            letters,
            probabilities,
            symbols,
            sum,
        }
    }
}
\end{lstlisting}
